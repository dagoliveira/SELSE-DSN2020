\section{Conclusions}
\label{sec_conclusion}

In this work, we show an initial experimental result to evaluate the differences between high energy and thermal neutron sensitivity. We evaluate modern computing devices used in domains such as HPC to consumer and autonomous vehicles. We show that thermal neutron sensitivity is far from negligible for most COTS devices, for instance, the DUE caused by thermal neutrons is as probable as the DUE caused by high energy neutrons
in the APU device.

We also discuss the neutron fluxes and how the surrounding material may increase the thermal neutron flux. We demonstrate through a thermal neutron detector that water from cooling systems can increase up to 20\% the thermal neutron flux. Finally, we report in~\cite{jsc2020} that the thermal neutron contribution to the total error rate can be up to 40\%.


%In this paper we have experimentally investigate the differences between high energy and thermal neutron induced error rates in modern HPC devices.  By irradiating devices  with high energy and thermal neutrons while executing representative applications,  we have demonstrated that thermals significantly impact device reliability.  We have demonstrated that the impact of high energy and thermal neutrons depends not only on the specifics of the hardware, but also on the executed code. The impinging neutron energy has more or less effect depending on how the code accesses memory and executes instructions. 
%
%We have also shown that the FIT rates can vary based on the physical layout of the machine room in which a system resides and variations such as weather conditions external to the building.  The reported data attests the importance of thermal neutron reliability evaluation, which can significantly raise the total device error rate. 
