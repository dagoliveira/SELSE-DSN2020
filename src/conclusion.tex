\section{Conclusions}
\label{sec_conclusion}

In this paper we have experimentally investigate the differences between high
energy and thermal neutron induced error rates in modern HPC devices. %While purifying the Silicon dopant to remove $^{10}B$ would make  devices immune to thermal neutrons, most COTS still use natural Boron. 
By irradiating devices  with high energy and thermal neutrons while executing representative applications,  we have demonstrated that thermals significantly impact device reliability.
We have demonstrated that the impact of high energy and thermal neutrons depends not only on the specifics of the hardware, but also on the executed code. The impinging neutron energy has more or less effect depending on how the code accesses memory and executes instructions. 

We have also shown that the FIT rates can vary based on the physical layout of the machine room in which a system resides and variations such as weather conditions external to the building. 
%We have also evaluate the FIT rates on different scenarios, considering concrete walls %attenuation and weather conditions. 
%When it is raining, the thermal-induced error rate significantly increases. 
%
The reported data attests the importance of thermal neutron reliability
evaluation, which can significantly raise the total device error rate. %As a future work, we plan to irradiate with thermal and high energy neutrons specific resources or components to deeply investigate different fault models. We also plan a thorough and sophisticated modeling of one or more  data centers as well as the effects of different cooling regimes.
