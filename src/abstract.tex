The low cost of Commercial-Off-The-Shelf (COTS) devices, as well as the high performance and efficiency, makes them an attractive option for applications with strict reliability constraints. Then, COTS devices are used in a wide range of domains, from consumer to HPC and autonomous vehicles. However, the manufacturing process of the COTS chip employs cheap natural Boron making COTS devices highly susceptible to errors induced by thermal neutrons.

We show, using radiation beam experiments with high energy and thermal neutrons, that the thermal neutrons threat is as significant as the high energy one. The evaluation includes devices used in several domains: an AMD APU, two NVIDIA GPUS, an Intel Xeon Phi, and an FPGA. Moreover, we discuss neutron flux, which is fundamental to measure the actual error rate and demonstrate, using a thermal neutron detector, that surrounding materials such as a concrete floor and water from cooling systems, significantly increases the thermal neutron flux.

%The high performance, high efficiency, and low cost of Commercial Off-The-Shelf (COTS) devices make them attractive for applications with strict reliability constraints. Today, COTS devices are adopted in HPC and safety-critical applications such as autonomous driving. Unfortunately, the cheap natural Boron widely used in COTS chip manufacturing process makes them highly susceptible to thermal (low energy) neutrons.
% 
%Through radiation beam experiments, using high-energy and low-energy neutrons, it has been shown that thermal neutrons are a significant threat to COTS device reliability. The evaluation includes AMD APU, three different NVIDIA GPUs, an Intel accelerator, and an FPGA executing a relevant set of algorithms. Besides the sensitivity of the devices to thermal neutrons it is also fundamental to consider the thermal neutron flux in different scenarios such as weather, concrete walls and floors, or even HPC liquid cooling systems. Correlating beam experiments and neutron detector data, it is shown that thermal neutrons FIT rate could be comparable or even higher than the high energy neutron FIT rate.

