\section{Related Work}
\label{sec:related}

HPC devices reliability has been widely studied using radiation experiments~\cite{Oliveira2017}, software fault injection~\cite{Wilkening2014, GPUQin, Tan2011}, or both strategies~\cite{Cher2014, SC16}. Radiation experiments provide a realistic device or application error rate~\cite{Baumann2005, ziegler2004ser}. Software fault injection, on the other hand, provides detailed information about faults propagation in the device architecture or code execution~\cite{AVF, PVF}.
Some studies also analyze field data from supercomputers error logs to draw conclusions about devices or system reliability~\cite{sridharan2013feng,sridharan2015}.
However, the impact of the thermal neutron on HPC devices and applications reliability has not been evaluated, yet.

Thermal neutrons have been known to cause errors in computing devices since the late 1990s and early 2000s~\cite{baumann1995boron,normand2006quantifying}. The investigation on the causes of unexpectedly high error rates of some devices shown that $^{10}B$ additives in the borophosphosilicate glass (BPSG), extensively used at the time as an insulating layer, increases the device error rate by $8\times$~\cite{baumann1995boron}. As the $^{10}B$ susceptibility to thermal neutrons became known, manufacturers removed BPSG entirely, and the thermal neutron issue was considered solved, having researchers focusing on other sources of errors.

Lately, $^{10}B$ was found in the manufacturing process of COTS devices~\cite{wen2010b10}. Some works have focused on SRAMs and FPGAs devices sensitivity to thermal neutrons~\cite{lee2015radiation,fang2016characterization,maillard2015neutron}. Weulersse et al.~\cite{weulersse2018contribution} compared the error rates of some memories (SRAM, CLB, caches) induced by thermal neutrons, 60MeV protons, and 14MeV neutrons. This preliminary study shows that the sensitivity to thermal neutrons ranges from $1.4x$ to $0.03x$ the high energy neutron one. While very interesting, experiments were conducted on just memory, and the work is not intended for HPC. Memory errors are less interesting for HPC devices as they can be efficiently masked or detected through ECC and parity. Unfortunately, Weulersse et al. do not share details about the kind of errors observed during their experiments (single vs. multiple bit flips), impeding to further evaluate the impact of their findings in HPC devices with ECC enabled. 

Our work advances the knowledge on HPC reliability by considering thermal neutrons impact on the reliability of HPC devices executing a representative set of applications. The radiation experiments are performed on devices executing representative applications under operative configurations (i.e., protection mechanisms enabled) to provide a realistic comparison between the error rates induced by high energy and thermal neutrons. Unlike previous publications, we perform both thermal and high energy neutrons experiments on exactly the same devices in the same operative conditions to limit comparison uncertainty. Furthermore, for the first time, we investigate through thermal neutrons detector, how modern supercomputer room and cooling systems designs influence the thermal neutron flux and, thus, the HPC devices error rates.
