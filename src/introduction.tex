\section{Introduction}
\label{sec_introduction}

Commercial Off-The-Shelf (COTS) devices are designed and produced to improve performances and efficiency at the lowest possible cost. Reliability is of  second importance, as long as the quality of service is sufficient for the average user. Building a device to the highest achievable levels of reliability conflicts with the low margins of the markets that consume these parts. Most consumers of COTS parts are primarily interested in performance and low price and would not understand a higher prize in exchange for higher reliability. The COTS part is suppose to be intrinsically sufficiently reliable, in other words, reliability is not seen a feature but an intrinsic characteristic~\cite{ziegler2004ser, Baumann2005}. 

On the other hand, reliability is one of the most important characteristics for devices in the field of High Performance Computing (HPC) and safety-critical applications~\cite{toptenexascale, NSF2018, ISO26262}.
As an unreliable system can negatively affect not only the throughput of a computer but also the correctness of operations, it is fundamental to increase reliability through redundancies in improved manufacturing processes, chip architectures, transistor layout changes, or other software/hardware hardening solutions~\cite{ziegler2004ser}. This added reliability clearly comes at an added cost in terms of engineering, more costly manufacturing, and increased power consumption. Only specialized safety critical industries, such as aerospace or medical, are today willing to pay the additional cost of highly
reliable parts. The HPC community is forced to build systems using COTS parts, because of its strict performance and power consumption constrains, forcing a constant struggle between the drive for ever increasing efficiency and the potential of lower scientific productivity due to lower reliability~\cite{snir2014addressing}. The automotive market is also peculiar, as reliability must be paramount (although not seen as a feature by the consumer) but the competitiveness of the market forces the production cost to be as low as possible. The rising interest of autonomous vehicles, that require extremely high computing capabilities, exacerbates this paradox.

In this paper we review the latest discoveries and the challenges associated with the reliability risk to HPC systems and safety-critical applications that comes from \textit{high energy} and \textit{thermal neutrons} generated. We refer in particular, to~\cite{ETS, jsc2020}, where authors propose an experimental evaluation of the sensitivity of various devices to high energy and thermal neutrons as well as an estimation of the error rates in various applications scenarios.
It is well known that $^{10}B$ has a large capture cross section for thermal neutrons and the resulting excited state of $^{10}B$ decays into Lithium-7 and a 1.47 MeV alpha particle~\cite{ziegler2004ser, Baumann2005}. This alpha particle, which has high energy, is known to be a threat for computing devices, as it can generate transient faults. It is just the presence of $^{10}B$ that increases significantly the susceptibility of the device to thermal neutrons, thus removing completely boron or using depleted $^{11}B$ would solve the problem and guarantee the device to be immune to thermal neutrons. However, this solution would significantly increase the manufacturing cost. Such a cost increase is unjustified for COTS devices, for which reliability is not as important as performances, efficiency, and yield. Using costly manufacturing process would make the device more reliable but most of the customers of COTS devices will not appreciate the difference and would not understand the increased product prize. As this COTS devices are today used in many applications for which reliability is a critical issue, such as HPC servers and automotive applications, it is fundamental to understand the error rate induced by also thermal neutrons, without assuming that $^{10}B$ is absent. The details of how  $^{10}B$ is used in modern chips is proprietary and not publicly available. The only way to evaluate boron concentration in a chip, and the associated increased sensitivity to thermal neutrons, is through controlled radiation exposure. We studied the effects of fast and thermal neutrons on modern computing devices executing a representative set of benchmarks.

As shown in~\cite{JSC}, modern data centers contain large masses of materials that can potentially increase the flux of thermal neutrons, in the form of concrete slab floors, cinder block walls, and water cooling units. In order to accurately estimate the effects of thermal neutrons we deployed a neutron detector to measure the natural background rate variation due to materials used in a modern data center. The measurements reported in~\cite{jsc2020} indicate that these materials  can increase the thermal neutron counts, and thus the COTS device's error rate, by as much as 20\%. 


%an AMD Accelerated Processing Unit (APU), three NVIDIA GPUs, an Intel accelerator,
%and a Xilinx Field Programmable Gate Array (FPGA) all executing a set of 8
%representative benchmarks that includes HPC applications, Convolutional Neural
%Networks (CNNs) for objects detection, and heterogeneous codes. 
We show that all the considered devices are vulnerable to thermal neutrons. For some devices, the probability for thermal neutrons to generate an error appears to be higher than the probability due to high energy neutrons.  

The main contributions of this paper are: (1) an experimental evaluation of the probability for a high energy vs. thermal neutron to generate an error in modern computing devices; (2) an estimation of the thermal neutrons flux modification due to materials heavily present in a supercomputer room, based on homemade thermal neutrons detectors; (3) the evaluation, based on (1) and (2), of the contribution of thermal neutrons to the error rate of computing devices. 


%The remainder of the paper is organized as follows. Section~\ref{sec_background} serves as a background and reviews previous work.
%Section~\ref{sec_methodology} describes our evaluation methodologies. Section~\ref{sec_results} quantifies our experimental results, Section~\ref{sec_fit} presents the estimated FIT rates, and Section~\ref{sec_conclusion} concludes the paper.

